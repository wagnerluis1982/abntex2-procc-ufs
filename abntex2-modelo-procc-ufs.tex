%% abtex2-modelo-trabalho-academico.tex, v-1.9.2 laurocesar
%% Copyright 2012-2014 by abnTeX2 group at http://abntex2.googlecode.com/
%%
%% This work may be distributed and/or modified under the
%% conditions of the LaTeX Project Public License, either version 1.3
%% of this license or (at your option) any later version.
%% The latest version of this license is in
%%   http://www.latex-project.org/lppl.txt
%% and version 1.3 or later is part of all distributions of LaTeX
%% version 2005/12/01 or later.
%%
%% This work has the LPPL maintenance status `maintained'.
%%
%% The Current Maintainer of this work is the abnTeX2 team, led
%% by Lauro César Araujo. Further information are available on
%% http://abntex2.googlecode.com/
%%
%% This work consists of the files abntex2-modelo-trabalho-academico.tex,
%% abntex2-modelo-include-comandos and abntex2-modelo-references.bib
%%

%% Este modelo foi adaptado para adequar-se ao modelo de Dissertação de Mestrado
%% do PROCC UFS por Wagner Macedo.
%%

% ------------------------------------------------------------------------
% ------------------------------------------------------------------------
% abnTeX2: Modelo de Trabalho Academico (tese de doutorado, dissertacao de
% mestrado e trabalhos monograficos em geral) em conformidade com
% ABNT NBR 14724:2011: Informacao e documentacao - Trabalhos academicos -
% Apresentacao
% ------------------------------------------------------------------------
% ------------------------------------------------------------------------

\PassOptionsToPackage{hyphens}{url}

\documentclass[
	% -- opções da classe memoir --
	12pt,				% tamanho da fonte
	oneside,			% para impressão apenas no anverso
	a4paper,			% tamanho do papel.
	% -- opções da classe abntex2 --
	%chapter=TITLE,		% títulos de capítulos convertidos em letras maiúsculas
	%section=TITLE,		% títulos de seções convertidos em letras maiúsculas
	%subsection=TITLE,	% títulos de subseções convertidos em letras maiúsculas
	%subsubsection=TITLE,% títulos de subsubseções convertidos em letras maiúsculas
	% -- opções do pacote babel --
	english,			% idioma adicional para hifenização
	brazil				% o último idioma é o principal do documento
	]{abntex2}

% ---
% Pacotes básicos
% ---
\usepackage{txfonts}			% Usa a fonte Times
\usepackage[T1]{fontenc}		% Selecao de codigos de fonte.
\usepackage[utf8]{inputenc}		% Codificacao do documento (conversão automática dos acentos)
\usepackage{lastpage}			% Usado pela Ficha catalográfica
\usepackage{indentfirst}		% Indenta o primeiro parágrafo de cada seção.
\usepackage{color}				% Controle das cores
\usepackage{graphicx}			% Inclusão de gráficos
\usepackage{microtype} 			% para melhorias de justificação

% ---
% Pacotes básicos - extra
% ---
\usepackage{caption}            % Melhor controle dos captions
\usepackage{subcaption}         % Inclusão de subfiguras

% Insere páginas PDF
\usepackage{pdfpages}
\newcommand{\includepaper}[1]{\includepdf[width=\paperwidth,height=!]{#1}}
% ---

% ---
% Pacotes adicionais, usados apenas no âmbito do Modelo Canônico do abnteX2
% ---
\usepackage{lipsum}				% para geração de dummy text
% ---

% ---
% Customizações para ficar de acordo com as normas da UFS
% ---
% Renomeada a seção "Lista de ilustrações"
\addto\captionsbrazil{%
  \renewcommand{\listfigurename}{Lista de figuras}%
}
% ---
% Fontes dos capítulos em serifa e negrito
\renewcommand{\ABNTEXchapterfont}{\rmfamily\bfseries}
% ---
% Formatação da Capa
\renewcommand{\imprimircapa}{%
  \begin{capa}%
    \center
    \bfseries\imprimirinstituicao

    \vfill
    \ABNTEXchapterfont\bfseries\LARGE\imprimirtitulo

    \vfill
    \ABNTEXchapterfont\large\imprimirautor

    \vfill
    \large\imprimirlocal\\
    \large\imprimirdata

    \vspace*{1cm}
  \end{capa}
}
% ---
% Formatação da Folha de Rosto
\makeatletter
\renewcommand{\folhaderostocontent}{
  \begin{center}
    {\bfseries\imprimirinstituicao}

    \vfill
    {\ABNTEXchapterfont\large\imprimirautor}

    \vfill
    \begin{center}
      \ABNTEXchapterfont\bfseries\Large\imprimirtitulo
    \end{center}

    \vfill
    \hfill
    \begin{minipage}{.53\textwidth}
      \SingleSpacing
       \imprimirpreambulo
     \end{minipage}%

    \vfill
    {\large\imprimirorientadorRotulo~\imprimirorientador\par}
    \abntex@ifnotempty{\imprimircoorientador}{%
       {\large\imprimircoorientadorRotulo~\imprimircoorientador}%
    }%

    \vfill
    {\large\imprimirlocal}\\
    {\large\imprimirdata}
    \vspace*{1cm}

  \end{center}
}
\makeatother

% ---
% Comando \url sem "<" e ">"
% ---
\DeclareUrlCommand\oNoLinkUrl{\def\UrlLeft{}\def\UrlRight{}\urlstyle{rm}}
\let\oHref\href % garante uso do \href sem futuras modificações
\let\oUrl\url   % guarda \url original
\renewcommand{\url}[1]{\oHref{#1}{\oNoLinkUrl{#1}}}

% ---
% Um \texttt menos problemático com as margens
% ---
\DeclareUrlCommand\oTextTT{\def\UrlLeft{}\def\UrlRight{}\urlstyle{tt}}
\let\texttt\oTextTT

% ---
% Outros pacotes e configurações
% ---

% permite verbatim em notas de rodapé
% ---
\usepackage{fancyvrb}
\VerbatimFootnotes
% tabelas
% ---
\usepackage{longtable,booktabs,tabularx}
\usepackage{afterpage}
% figuras
% ---
\usepackage{graphicx,grffile}
\makeatletter
\def\maxwidth{\ifdim\Gin@nat@width>\linewidth\linewidth\else\Gin@nat@width\fi}
\def\maxheight{\ifdim\Gin@nat@height>\textheight\textheight\else\Gin@nat@height\fi}
\makeatother
% Scale images if necessary, so that they will not overflow the page
% margins by default, and it is still possible to overwrite the defaults
% using explicit options in \includegraphics[width, height, ...]{}
% ---
\setkeys{Gin}{width=\maxwidth,height=\maxheight,keepaspectratio}
% tightlist
% ---
\providecommand{\tightlist}{%
  \setlength{\itemsep}{0pt}\setlength{\parskip}{0pt}}

% ---
% Único contador para equações matemáticas
% ---
\counterwithout{equation}{chapter}

% ---
% Numeracao das figuras e tabelas por capítulo
% ---
\counterwithin{figure}{chapter}
\counterwithin{table}{chapter}
% ---

% ---
% Pacotes de citações
% ---
% Citações padrão ABNT
\usepackage[alf]{abntex2cite}

% use upquote if available, for straight quotes in verbatim environments
\IfFileExists{upquote.sty}{\usepackage{upquote}}{}

% ---
% CONFIGURAÇÕES DE PACOTES
% ---

% ---
% Informações de dados para CAPA e FOLHA DE ROSTO
% ---
\titulo{Modelo Não Oficial de Dissertação de Mestrado do PROCC~UFS}
\autor{Wagner Macedo}
\local{São Cristóvão}
\data{2016}
\orientador{Prof. Dr. Fulano de Tal}
\coorientador{Prof. Dr. Ciclano da Silva}
\instituicao{\uppercase{%
  Universidade Federal de Sergipe
  \par
  Centro de Ciências Exatas e Tecnológicas
  \par
  Programa de Pós-graduação em Ciência da Computação}}
\tipotrabalho{Dissertação (Mestrado)}
% O preambulo deve conter o tipo do trabalho, o objetivo,
% o nome da instituição e a área de concentração
\preambulo{Dissertação apresentada ao Programa de Pós-Graduação em Ciência da
Computação (PROCC) da Universidade Federal de Sergipe (UFS) como parte
de requisito para obtenção do título de Mestre em Ciência da Computação.}
% ---

% ---
% Configurações de aparência do PDF final

% alterando o aspecto da cor azul
\definecolor{blue}{RGB}{41,5,195}

% informações do PDF
\makeatletter
\hypersetup{
     	%pagebackref=true,
		pdftitle={\@title},
		pdfauthor={\@author},
    	pdfsubject={\imprimirpreambulo},
	    pdfcreator={LaTeX with abnTeX2},
		pdfkeywords={abnt, latex, abntex, abntex2, disseração},
		pdfborder=0 0 0,
		colorlinks=false,       	% false: boxed links; true: colored links
    	linkcolor=blue,          	% color of internal links
    	citecolor=blue,        		% color of links to bibliography
    	filecolor=magenta,      		% color of file links
		urlcolor=blue,
		bookmarksdepth=4
}
\makeatother
% ---

% ---
% Espaçamentos entre linhas e parágrafos
% ---

% O tamanho do parágrafo é dado por:
\setlength{\parindent}{1.3cm}

% Controle do espaçamento entre um parágrafo e outro:
\setlength{\parskip}{0.2cm}  % tente também \onelineskip

% ---
% compila o indice
% ---
\makeindex
% ---

% Hyphenation
\hyphenation{}

% ----
% Início do documento
% ----
\begin{document}

% Retira espaço extra obsoleto entre as frases.
\frenchspacing

% ----------------------------------------------------------
% ELEMENTOS PRÉ-TEXTUAIS
% ----------------------------------------------------------
% \pretextual

% ---
% Capa
% ---
\imprimircapa
% ---

% ---
% Folha de rosto
% (o * indica que haverá a ficha bibliográfica)
% ---
\imprimirfolhaderosto*
% ---

% ---
% Inserir a ficha bibliografica
% ---

% Isto é um exemplo de Ficha Catalográfica, ou ``Dados internacionais de
% catalogação-na-publicação''. Você pode utilizar este modelo como referência.
% Porém, a Biblioteca Central da UFS lhe fornecerá um PDF com a ficha
% catalográfica definitiva após a defesa do trabalho. Quando estiver com o
% documento, salve-o como PDF no diretório do seu projeto e substitua todo o
% conteúdo de implementação da ficha pelo comando abaixo:
%
% \begin{fichacatalografica}
%     \includepaper{ficha-catalografica-bicen.pdf}
% \end{fichacatalografica}
\begin{fichacatalografica}
	\vspace*{\fill}					% Posição vertical
	\hrule							% Linha horizontal
	\begin{center}					% Minipage Centralizado
	\begin{minipage}[c]{12.5cm}		% Largura

	\imprimirautor

	\hspace{0.5cm} \imprimirtitulo  / \imprimirautor. --
	\imprimirlocal, \imprimirdata-

	\hspace{0.5cm} \pageref{LastPage} p. : il. (algumas color.) ; 30 cm.\\

	\hspace{0.5cm} \imprimirorientadorRotulo~\imprimirorientador\\

	\hspace{0.5cm}
	\parbox[t]{\textwidth}{\imprimirtipotrabalho~--~\imprimirinstituicao,
	\imprimirdata.}\\

	\hspace{0.5cm}
		1. Palavra-chave1.
		2. Palavra-chave2.
		I. Orientador.
		II. Universidade xxx.
		III. Faculdade de xxx.
		IV. Título\\

	\hspace{8.75cm} CDU 02:141:005.7\\

	\end{minipage}
	\end{center}
	\hrule
\end{fichacatalografica}
% ---

% ---
% Inserir banca examinadora
% ---
\begin{folhadeaprovacao}[Banca Examinadora]

  \begin{center}
    {\ABNTEXchapterfont\large\imprimirautor}

    \vspace*{\fill}
    \begin{center}
      \ABNTEXchapterfont\bfseries\Large\imprimirtitulo
    \end{center}
    \vspace*{\fill}

    \hfill
    \begin{minipage}{.53\textwidth}
        \imprimirpreambulo
    \end{minipage}%
    \vspace*{\fill}
   \end{center}

  \begin{center}
    \ABNTEXchapterfont\bfseries\large{\uppercase{Banca Examinadora}}
    \vspace*{-1cm}
  \end{center}

   \setlength{\ABNTEXsignwidth}{10cm}
   \setlength{\ABNTEXsignthickness}{0pt}
   \assinatura{%
      \imprimirorientador, Presidente\\%
        Universidade Federal de Sergipe (UFS)}
      \assinatura{Prof. Dr. Zé Ninguém, Membro\\%
        Universidade Federal de Sergipe (UFS)}
      \assinatura{Prof. Dr. Clark Kent, Membro\\%
      Universidade de Krypton (UK)}

   \vspace*{\fill}\vspace*{\fill}
\end{folhadeaprovacao}
% ---

% ---
% Inserir folha de aprovação
% ---

% Isto é um exemplo de Folha de aprovação, elemento obrigatório da NBR
% 14724/2011 (seção 4.2.1.3). Você pode utilizar este modelo até a aprovação
% do trabalho. Após isso, substitua todo o conteúdo deste arquivo por uma
% imagem da página assinada pela banca com o comando abaixo:
%
% \includepaper{folhadeaprovacao_final.pdf}
%
\begin{folhadeaprovacao}
  \begin{center}
    \begin{center}
      \ABNTEXchapterfont\bfseries\LARGE\imprimirtitulo
    \end{center}

    \vfill
    \hfill
    \begin{minipage}{.51\textwidth}
      Este exemplar corresponde à redação da Dissertação de Mestrado, sendo a
      defesa do mestrando \textbf{\imprimirautor} para ser aprovada pela banca
      examinadora.
    \end{minipage}%
  \end{center}

  \vfill
  \begin{center}
    Trabalho aprovado. \imprimirlocal, 29 de fevereiro de 2016:
  \end{center}

   \setlength{\ABNTEXsignwidth}{9cm}
   \assinatura{\textbf{\imprimirorientador}\\Orientador}
   \assinatura{\textbf{Prof. Dr. Ciclano da Silva}\\Membro}
   \assinatura{\textbf{Prof. Dr. Beltrano dos Santos}\\Membro}

   \vfill
   \vfill

\end{folhadeaprovacao}
% ---

% ---
% Dedicatória
% ---
\begin{dedicatoria}
   \vspace*{\fill}
   \centering
   \noindent
   \textit{Este trabalho é dedicado àqueles\\
           que amam compartilhar.} \vspace*{\fill}

\end{dedicatoria}
% ---

% ---
% Agradecimentos
% ---
\begin{agradecimentos}
Seus agradecimentos...
\end{agradecimentos}
% ---

% ---
% RESUMOS
% ---

% resumo em português
\setlength{\absparsep}{18pt} % ajusta o espaçamento dos parágrafos do resumo
\begin{resumo}
 Segundo a \citeonline[3.1-3.2]{NBR6028:2003}, o resumo deve ressaltar o
 objetivo, o método, os resultados e as conclusões do documento. A ordem e a extensão
 destes itens dependem do tipo de resumo (informativo ou indicativo) e do
 tratamento que cada item recebe no documento original. O resumo deve ser
 precedido da referência do documento, com exceção do resumo inserido no
 próprio documento. (\ldots) As palavras-chave devem figurar logo abaixo do
 resumo, antecedidas da expressão Palavras-chave:, separadas entre si por
 ponto e finalizadas também por ponto.

 \noindent
 \textbf{Palavras-chaves}: latex, abntex, editoração de texto.
\end{resumo}

% resumo em inglês
\begin{resumo}[Abstract]
 \begin{otherlanguage*}{english}
   This is the english abstract.

   \noindent
   \textbf{Key-words}: latex, abntex, text editoration.
 \end{otherlanguage*}
\end{resumo}

% ---
% inserir lista de figuras
% ---
\pdfbookmark[0]{\listfigurename}{lof}
\listoffigures*
\cleardoublepage
% ---

% ---
% inserir lista de tabelas
% ---
\pdfbookmark[0]{\listtablename}{lot}
\listoftables*
\cleardoublepage
% ---

% ---
% inserir lista de abreviaturas e siglas
% ---
\begin{siglas}
  \item[ABNT] Associação Brasileira de Normas Técnicas
  \item[abnTeX] ABsurdas Normas para TeX
\end{siglas}
% ---

% ---
% inserir o sumario
% ---
\pdfbookmark[0]{\contentsname}{toc}
\tableofcontents*
\cleardoublepage
% ---


% ----------------------------------------------------------
% ELEMENTOS TEXTUAIS
% ----------------------------------------------------------
\textual

% ----------------------------------------------------------
% Introdução (exemplo de capítulo sem numeração, mas presente no Sumário)
% ----------------------------------------------------------
\chapter{Introdução}
% ----------------------------------------------------------

Este documento e seu código-fonte são exemplos de referência de uso da classe
\textsf{abntex2} e do pacote \textsf{abntex2cite}. O documento
exemplifica a elaboração de trabalho acadêmico (tese, dissertação e outros do
gênero) produzido conforme a ABNT NBR 14724:2011 \emph{Informação e documentação
- Trabalhos acadêmicos - Apresentação}.

A expressão ``Modelo Canônico'' é utilizada para indicar que \abnTeX\ não é
modelo específico de nenhuma universidade ou instituição, mas que implementa tão
somente os requisitos das normas da ABNT. Uma lista completa das normas
observadas pelo \abnTeX\ é apresentada em \citeonline{abntex2classe}.

Sinta-se convidado a participar do projeto \abnTeX! Acesse o site do projeto em
\url{http://abntex2.googlecode.com/}. Também fique livre para conhecer,
estudar, alterar e redistribuir o trabalho do \abnTeX, desde que os arquivos
modificados tenham seus nomes alterados e que os créditos sejam dados aos
autores originais, nos termos da ``The \LaTeX\ Project Public
License''\footnote{\url{http://www.latex-project.org/lppl.txt}}.

Encorajamos que sejam realizadas customizações específicas deste exemplo para
universidades e outras instituições --- como capas, folha de aprovação, etc.
Porém, recomendamos que ao invés de se alterar diretamente os arquivos do
\abnTeX, distribua-se arquivos com as respectivas customizações.
Isso permite que futuras versões do \abnTeX~não se tornem automaticamente
incompatíveis com as customizações promovidas. Consulte
\citeonline{abntex2-wiki-como-customizar} par mais informações.

Este documento deve ser utilizado como complemento dos manuais do \abnTeX\
\cite{abntex2classe,abntex2cite,abntex2cite-alf} e da classe \textsf{memoir}
\cite{memoir}.

Esperamos, sinceramente, que o \abnTeX\ aprimore a qualidade do trabalho que
você produzirá, de modo que o principal esforço seja concentrado no principal:
na contribuição científica.

Equipe \abnTeX

Lauro César Araujo


% ---
% Capitulo com exemplos de comandos inseridos de arquivo externo
% ---
\include{include-comandos}
% ---


% ---
% Capitulo de revisão de literatura
% ---
\chapter{Lorem ipsum dolor sit amet}
% ---

% ---
\section{Aliquam vestibulum fringilla lorem}
% ---

\lipsum[1]

\lipsum[2-3]


% ---
% primeiro capitulo de Resultados
% ---
\chapter{Lectus lobortis condimentum}
% ---

% ---
\section{Vestibulum ante ipsum primis in faucibus orci luctus et ultrices
posuere cubilia Curae}
% ---

\lipsum[21-22]

% ---
% segundo capitulo de Resultados
% ---
\chapter{Nam sed tellus sit amet lectus urna ullamcorper tristique interdum
elementum}
% ---

% ---
\section{Pellentesque sit amet pede ac sem eleifend consectetuer}
% ---

\lipsum[24]

% ----------------------------------------------------------
% Finaliza a parte no bookmark do PDF
% para que se inicie o bookmark na raiz
% e adiciona espaço de parte no Sumário
% ----------------------------------------------------------
\phantompart

% ---
% Conclusão
% ---
\chapter{Conclusão}
% ---

\lipsum[31-33]

% ----------------------------------------------------------
% ELEMENTOS PÓS-TEXTUAIS
% ----------------------------------------------------------
\postextual
% ----------------------------------------------------------

% ----------------------------------------------------------
% Referências bibliográficas
% ----------------------------------------------------------
\let\url\oUrl
\bibliography{references}

% ----------------------------------------------------------
% Glossário
% ----------------------------------------------------------
%
% Consulte o manual da classe abntex2 para orientações sobre o glossário.
%
%\glossary

% ----------------------------------------------------------
% Apêndices
% ----------------------------------------------------------

% ---
% Inicia os apêndices
% ---
\begin{apendicesenv}

% Imprime uma página indicando o início dos apêndices
\partapendices

% ----------------------------------------------------------
\chapter{Quisque libero justo}
% ----------------------------------------------------------

\lipsum[50]

% ----------------------------------------------------------
\chapter{Nullam elementum urna vel imperdiet sodales elit ipsum pharetra ligula
ac pretium ante justo a nulla curabitur tristique arcu eu metus}
% ----------------------------------------------------------
\lipsum[55-57]

\end{apendicesenv}
% ---


% ----------------------------------------------------------
% Anexos
% ----------------------------------------------------------

% ---
% Inicia os anexos
% ---
\begin{anexosenv}

% Imprime uma página indicando o início dos anexos
\partanexos

% ---
\chapter{Morbi ultrices rutrum lorem.}
% ---
\lipsum[30]

% ---
\chapter{Cras non urna sed feugiat cum sociis natoque penatibus et magnis dis
parturient montes nascetur ridiculus mus}
% ---

\lipsum[31]

% ---
\chapter{Fusce facilisis lacinia dui}
% ---

\lipsum[32]

\end{anexosenv}

\end{document}
